\documentclass{article}
\usepackage[utf8]{inputenc}
\usepackage{multicol}
% maxamize space on paper
\usepackage[nomarginpar, margin=.1in]{geometry}
\usepackage{sectsty}
\usepackage{amsmath}
\sectionfont{\fontsize{9}{9}\selectfont}
\subsectionfont{\fontsize{9}{9}\selectfont}
\subsubsectionfont{\fontsize{9}{9}\selectfont}
\usepackage[compact]{titlesec}
\usepackage{enumitem}
\setlength{\parindent}{0pt} 
\setlist[1]{itemsep=-5pt}
\setlist[2]{itemsep=-5pt}
\begin{document}
\underline{This document is currently incomplete - Information may be incorrect - Please contribute on GitHub!}

\section{P, NP, NP-Completness}

\underline{P:} Generally refers to problems that can be solved in polynomial time

\underline{NP:} Generally refers to problems that can have their solution verified in polynomial time

\underline{NP-Complete:} A problem in NP in which a polynomial-time algorithm that can reduce
the it into any other NP-Complete problem. 

\underline{NP-Hard:} A set of problems that are hard to solve and verify, with some problems not being decideable. Reducable to any problem in NP

\underline{P vs NP:} The P vs NP problem is an unsolved problem on wether a problem whose solution can be
verified in polynomial time can also be solved in polynomial time. 

\underline{Proving NP-Complete via reduction:} If we wanted to prove that the Undirected Hamiltonian Path Problem 
is NP-Complete, we would reduce the Directed Hamiltonian Path Problem (which is NP-Complete) to it.

\underline{Proving NP-Hardvia reduction:} If we wanted to prove that the Undirected Hamiltonian Path Problem 
is NP-Complete, we would reduce the Directed Hamiltonian Path Problem (which is NP-Complete) to it.

\section{Graph Coloring}
Given a graph $G = (V, E)$, find the smallest number of different colors to assign for 
each nod ein $G$ so that no two nodes of the same color share an edge. Decision version: 
Given a graph $G$ and a bound $k$, does $G$ have k-coloring?

Simple when $k=2$ since you just need to check if $G$ is bipartite

When $k=3$, we must prove that it is NP-Complete by reducing it to 3-SAT (2-SAT $\leq_p$ 3-Coloring)

\section{Approximation}
\underline{Linear Programming:} Given a set of inequalities that represent constraints, our
goal is to minimize or maxamize a certain quantity represented 
by an equation.
$N<1$-approximations is typically used to represent maxamization problems while
an $N>1$-approximation is typically used to represent a minimization problem. 
$N$-approximation means that the solution is within $OPT*N$

In order to do this, find the worst case and then find a way to relate it to the best case.

\subsection{Load Balancing}
This problem is an example for an approx. algirhthm. Given $M$ machines $m_1, m_2, ..., m_n$ and $n$ jobs where each job $j$ has a processing time
$t_j$, assign each job to a machine and balance the loads across all machines.

\subsection{Vertex Cover}
The vertex cover problem is a problem in which you want to find a set $S$ of verticies in a graph such that
every edge connects to one of the verticies in $S$.

\subsection{Set Cover}
The set cover problem is a problem in which you are given a set $M$ of sets and a set $U$ of numbers. Your
goall is to find a union of some sets in $M$ such that their union equals $U$

\section{PSPACE and PSPACE-Completeness}
PSPACE is the set of decision problems that can be solved by a Turing Machine with a polynomial amount of
space. 

PSPACE-Completeness refers to a PSPACE problem in which every other PSPACE problem could be transformed to it in 
polynomial time

\section{Turing Machines}


\section{Undecidability and Halting problem}


\section{Randomized Algorithms}

\section{Misc}



\end{document}
